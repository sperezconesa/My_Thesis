\chapter[Conclusions]{Conclusions}\label{c5:conclusion}
In this thesis we have presented a set of \textit{ab initio} HIM force fields for 
\ce{[AnO2*(H2O)5]^{2+}} for An=U, Np, Pu, Am in water as well as an additional one for the 
interaction 
of 
uranyl with montmorillonite clay. These interaction potentials offer an alternative to current 
ones in particular since they pa\-ra\-me\-tri\-ze the HI-bulk water interactions. They have 
proven to reproduce satisfactorily many experimental properties of the systems: XAS spectra, 
hydration enthalpies, diffusion coefficients...

The developed potentials allowed the detailed study of the solvation of the actinyls. Their 
solvation 
was found to be amphiphillic and anisotropic which is remarkable considering the small size of the 
ion 
and its charge. From the MD simulations we calculated the XAS spectra of the 
actinyls and compared then to experiment. This allowed us to assess the quality of their structural 
model and to interpret the keys of their complex spectra. In particular, we 
interpreted the experimental XAS spectra of an \ce{Am^{3+}}/\ce{[AmO2]^{2+}} mixture 
reproducing the 
theoretical EXAFS spectrum of a mixture. With the experimental validation of our 
simulation we were able to predict the structural parameters and the EXAFS and XANES spectra of a 
pure 
\ce{[AmO2*(H2O)5]^{2+}} solution, a solution that has never been obtained. This will be a 
striking experimental challenge for actinoid solution chemistry in the near future.

The uranyl montmorillonite clay simulations allowed us to study the diffusion of the HI in the 
interlayers. We identified the existence of strong interaction sites for uranyl. These sites force 
the uranyl to diffuse following a hopping mechanism. Because of this,
uranyl diffusion increases with uranyl concentration due to cation-cation interactions 
and a larger coverage of 
surface sites.

Another methodological achievement of this thesis has been the development of a simple local 
fingerprint for hydrophobicity and hydrophilicity. The fingerprint was able to classify 
correctly 
the atoms of amino acids in water which have varied functional groups and chemical 
environments. The fingerprint has also been proven to be a useful solvation/desolvation CV in 
enhanced sampling simulations. When applied to characterize the different hydration regions 
around actinyls, fingerprint values do not seem to provide such a clear view as in the amino 
acid case. This opens the door to refinement of the index when dealing with metal cations and 
ions in general.
 

In conclusion, this thesis provides a different perspective to the study of aqueous actinyls and 
actinyls in clay interlayers. This different perspective is based on the usage of newly developed 
methodologies that enable a cost effective and fairly accurate view of these important systems. 
