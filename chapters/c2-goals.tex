\selectlanguage{english}
%%%%%%%%%%% Capitulo 2 Metodologia %%%%%%%%%%%%%%%%%%%%%%%%%%%%%%%%%%%%%%%%%%%%%%%%%%%%%%%
%\setcounter{chapter}{3}
\chapter[Thesis Goals]{Thesis Goals}\label{c:Goals}
The goals of this thesis were mostly planned but with some interesting 
surprises, new paths, tangents and occasional dead ends found during the course of research. 
The goals and the articles are not presented in a chronological order of publication nor in order 
of investigation, but following a scientifically consistent criterion. In research it is quite 
frequent to run projects in parallel, 
at other times drop all projects to focus on a single goal and all kinds of non-sequential 
work habits.

We had two general and main goals in mind at the beginning of  this project. The first goal was to 
analyze and understand
systems of relevance in radioactive chemistry using computational chemistry in some cases 
interfacing with experiment. The second goal was to fill the missing gaps in computational 
modelling that would allow the achievement of the aforementioned goals. These main purposes 
will take 
definite form as explained below.

The first step of this Ph. D. thesis was to study the physico-chemical properties of the 
\ce{[AnO2]^{2+/+}} species with An=U,Np,Pu,Am in dilute aqueous solution and inserted in clay 
interlayers. A wide set of physico-chemical properties have been determined computationally as well 
as a detailed analysis of the hydration properties and X-Ray Absorption Spectroscopy of these 
cations in condensed media. In the clay simulations we will focus on diffusion 
and interactions between the hydrated ions and the clay surfaces. The methodological 
achievements will demand the development of  \textit{ab initio} force fields based on the 
hydrated ion model (\gls{him}) that reproduce adequately the available experimental properties. 
An 
additional goal is to give insight into the synergic experimental-theoretical procedures to analyze 
involved XAS 
problems. 

The individual goals undertaken can also be associated to the different articles:
\begin{enumerate}
 \item \textbf{``A hydrated ion model of \ce{[UO2]^{2+}} in water: Structure, dynamics, and 
spectroscopy from classical molecular dynamics'':}\newline
To develop an \textit{ab initio} HIM force field for \ce{[UO2]^{2+}}-\ce{H2O} in water. To 
study 
its physico-chemical properties including structural, dynamical and spectroscopical 
properties. To characterize the solvation structure of uranyl.
 \item \textbf{``A general study of actinyl hydration by molecular dynamics simulations using 
ab initio force fields'':}\newline
To extend the methodology used to develop the force field for \ce{[UO2]^{2+}}-\ce{H2O} to 
the 
rest of the actinyls examining the partial transferability of the uranyl potential to the rest 
of 
actinyls. To 
study how the properties evolve in the series and the significance of the charge change
from divalent to monovalent.
 \item \textbf{``Extracting the Americyl Hydration from an Americium Cationic Mixture in 
Solution: A Combined X-ray Absorption Spectroscopy and Mo\-le\-cu\-lar Dynamics 
Study'':}\newline
To interpret the experimental XAS of an \ce{Am^{3+}}/\ce{[AmO2]^{2+}} aqueous mixture and clarify 
the structural parameters fitted by the experimentalists. To generate the theoretical 
XAS spectrum of the mixture of species. To use the \ce{[AmO2]^{2+}} contribution to the  
theoretical 
spectrum to predict the properties of the so far not measured pure americyl in aqueous solution. 
 \item \textbf{``Combining EXAFS and Computer Simulations to
Refine the Structural Description of Actinyl in Water'':} To generate and analyze theoretical EXAFS 
spectra of the actinyls using the trajectories of developed in the second article. To study the 
effect of tiny 
structural changes in the spectrum and analyze the importance of including 
higher levels of 
theory to study EXAFS spectra.
 \item \textbf{``Hydration and Diffusion Mechanism of Uranyl in Montmorillonite Clay: 
Molecular Dynamics Using an Ab Initio Potential'':}\newline
To develop an \textit{ab initio} HIM force field between the uranyl hydrated  ion  and at 
the clay
surface. To study the molecular dynamics of uranyl in montmorillonite clay interlayers and 
particularly the factors affecting its diffusion.
 \item \textbf{``A local fingerprint for hydrophobicity and hydrophilicity: from 
me\-tha\-ne to 
peptides'':}\newline
To develop a simple local fingerprint that measures the hydrophobicity and hydrophilicity of 
atoms in complex solutes and can serve as a collective variable in enhanced sampling 
simulations. To use the fingerprint as a desolvation collective variable in  metadynamics 
simulations. To apply the fingerprint to analyze the anisotropic hydration structure of actinyls.
\end{enumerate}
