\section[Hydrophobicity fingerprint of actinyls]{Hydrophobicity hydrophilicity 
fingerprint of actinyls.}

\begin{table}
\center
\caption[Fingerprint values for actinyls]{Hydrophobicity/hydrophilicity fingerprint values of 
actinyl atoms obtained from the 
simulations of Chapter \ref{art2}.}\label{tableAn}
\begin{tabular}{lccc}
\toprule
&$h$&$h^\text{HI}$&$h^{\text{HI}}_{\text{0-\SI{90}{\degree}}}$\\
\midrule
An       &-7.9   &  -5.2 & -  \\
\oyl  & -1.9 &   0.4 & -2.1 \\
\ofs  &0.8&   1.5  &0.5\\
\ce{H2O}    & 1.0&  1.0 & - \\
\ce{CH4}    &-1.0& -1.0 & -  \\
\bottomrule
\end{tabular}
\end{table}

Given the amphiphilic nature of the solvation of actinyls, the simulations carried out in 
Chapters \ref{art1} and \ref{art2} are perfect candidates to study the fingerprint behavior in 
complex cationic systems. We present the fingerprints of the heavy atoms of uranyl as  
representative of the rest of actinyls since due to their similarity in solvation the fingerprint 
values must be very similar even for \ce{[NpO2]^+}. The 
reference values of $S_s$ were calculated from a pure TIP4P-water simulation and a methane 
TIP4P-water simulation. Three versions of the fingerprint were examined:
\begin{itemize}
 \item $h$: uses the RDF between the atom of interest 
and all water molecules.
\item $h^\text{HI}$: uses the RDF between the fingerprinted atom and all bulk water molecules 
excluding 
the first-shell.
  \item $h^\text{HI}_\text{0-\SI{90}{\degree}}$: uses the 0-\SI{90}{\degree} angle-solved RDF 
between 
the fingerprinted atom and all bulk water molecules. The 0-\SI{90}{\degree} cone is calculated 
in the same fashion as in the angle-resolved RDFs of the \oyl atom in Chapter \ref{art1}.
\end{itemize}

The last two definitions connect with the HIM philosophy by considering first-shell water 
molecules part of the solute and not of the solvent. The fingerprint values are 
collected in Table 
\ref{tableAn}.

The first shell water molecules (\ofs) are correctly labeled as hydrophilic by the fingerprint in 
all 
versions. In contrast, the \oyl atoms are labeled as hydrophilic by $h^\text{HI}$ and 
hydrophobic by $h$. $h^\text{HI}$ is conceptually a better fingerprint since first-shell water 
molecules do not solvate the \oyl atom. Unfortunately, it also misclassifies the \oyl 
atom as hydrophilic.

As we have seen in Chapter \ref{art1}, the solvation of the uranyl atoms is 
very 
anisotropic and standard RDF  can lead to wrong conclusions 
i. e. the \oyl RDF 
includes solvent 
molecules that are in the bridge solvation region and are not solvating the \oyl atom. For 
this reason we decided to use as the fingerprint 
input the 0-\SI{90}{\degree} angle-solved RDF. In this way, the solvent molecules included in the 
RDF are the ones that solvate the atom. $h^\text{HI}_\text{0-\SI{90}{\degree}}$ classifies 
correctly 
the \oyl and \ofs atoms but it considers \ofs to be less hydrophilic than 
bulk water molecules.

The fingerprint classifies the actinoid cation as hydrophobic which is clear\-ly an artifact. 
In 
Section \ref{art5_sec1} this effect was also described for \ce{Na^+}. The order imposed on the 
solvent structure by the 
doubly charged cation is very high and thus obtaining $h$ values much lower than methane. 

In conclusion, the fingerprint appears to be unfit to classify the central atoms of cations 
with charge higher or equal to one. For the rest of the heavy atoms of the hydrated ion mixed 
results are obtained.

The fingerprint is too coarse for complex solutes like actinyls. A future improved 
fingerprint could probably make use of orientational pair entropy in addition to some technique to 
consider the anisotropicity of the solute in complex environments. 
